\documentclass[aspectratio=169]{beamer}

%=======================
%       IMPORTS
%=======================

\usepackage{amsmath, amsthm, amssymb}   % Math symbols, environments etc.
\usepackage{fontspec}                   % For Unicode fonts
\usepackage{unicode-math}               % For Unicode math fonts
\usepackage{microtype}                  % Nicer interword spacing
\usepackage{polyglossia}                % Provides locale-specific formatting
\usepackage{csquotes}
\usepackage{caption}                    % Provides captions outside of floats
\usepackage{svg}                        % For including SVG files
\usepackage{booktabs}                   % Nicer tables
\usepackage{tabularx}                   % Allows for linebreaks in tables

\usepackage[
    backend = biber,
    style = ieee-alphabetic
]{biblatex}

% Set beamer theme
\usetheme{metropolis}
% \beamertemplatenavigationsymbolsempty
\metroset
{
    sectionpage = simple,
    subsectionpage = simple
}

\usecolortheme{rose}

\setmathfont{TeX Gyre Pagella Math}

% Captions
\captionsetup{justification=centering}

% Language Setup
\setdefaultlanguage[spelling=new, babelshorthands=true]{german}

% Set path for graphics
\graphicspath{{./graphics/}}

% hyperref setup
\hypersetup
{
    pdfencoding = auto      % For umlauts in PDF sections
}

%=======================
%    CUSTOM COMMANDS
%=======================
\newcommand{\Ham}{\ensuremath{\mathbb{H}}}
\newcommand{\R}{\ensuremath{\mathbb{R}}{ }}

%=======================
%  DOCUMENT INFORMATION
%=======================

\title{\textsc{Hamilton}sche Quaternionen}
\subtitle{Proseminar Mathematik}
\author[L.~Richardt]{Leon Richardt}
\date[2020-07-07]{7. Juli 2020}
\institute{Universität Osnabrück}

\begin{document}
    \begin{frame}
        \titlepage
    \end{frame}

    \begin{frame}{Überblick}
        \tableofcontents
    \end{frame}

    \section{Reelle Algebren}
    \begin{frame}
        \begin{block}{Anmerkung}
            In dieser Präsentation stehen kleine griechische Buchstaben stets für reelle Zahlen; lateinische Buchstaben stehen für Elemente der momentan betrachteten Algebra.
        \end{block}
    \end{frame}

    \begin{frame}
        \begin{definition}
            Ein Vektorraum \(V\) über \R mit einer Produktabbildung
            \[
                V \times V \to V, (x, y) \mapsto xy
            \] 
            heißt \textbf{Algebra} über \R (oder reelle Algebra), wenn die beiden Distributivgesetze
            \begin{gather*}
                (\alpha x + \beta y) z = \alpha \cdot xz + \beta \cdot yz, \\
                x (\alpha y + \beta z) = \alpha \cdot xy + \beta \cdot xz
            \end{gather*}
            für alle \(\alpha, \beta \in \R\) und \(x, y, z \in V\) erfüllt sind.
        \end{definition}
    \end{frame}

    \begin{frame}
        \begin{definition}
            Eine Algebra heißt \dots
            \begin{itemize}
                \item
                    \dots{} \textit{assoziativ}, wenn \(x(yz) = (xy)z\) für alle \(x, y, z \in V\) gilt.
                \item
                    \dots{} \textit{kommutativ}, wenn \(xy = xy\) für alle \(x, y \in V\) gilt.
                \item
                    \dots{} \textit{mit Einselement}, wenn es ein Element \(e \in V\) mit \(ex = xe = x\) für alle \(x \in V\) gibt.
            \end{itemize}
        \end{definition}
    \end{frame}

    \begin{frame}
        \begin{definition}
            Ein Element \(x\) einer Algebra \(\mathcal{A}\) heißt \textit{Nullteiler}, falls es ein Element \(0 \neq y \in \mathcal{A}\) mit \(xy = 0\) oder \(yx = 0\) gibt.

            Konsequenterweise heißt eine Algebra \textit{nullteilerfrei}, falls sie keine Nullteiler \(\neq 0\) besitzt.
        \end{definition}
    \end{frame}

    \section{Historisches}

    \section{Die Quaternionenalgebra \(\mathbb{H}\)}
    \section{Der Imaginärraum von \(\mathbb{H}\)}
    \subsection{Bezug zu klassischen Vektorprodukten}
    \section{Zentrum von \(\mathbb{H}\)}
    \section{Endomorphismen von \(\mathbb{H}\)}
    \section{Fundamentalsatz der Algebra für Quaternionen}
\end{document}
